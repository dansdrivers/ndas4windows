
@@Overview
<B>Purpose</B>

NDAS Communication API(NDASComm) and NDAS Heartbeat
Listener(NDASHear) API enable programmers to operate a NDAS
Device. With NDASComm and NDASHear, programmers can operate
not only basic operations such as reading or writing, but
also advanced operations including most parts of the ATA
commands such as identifying or setting features.

<B>

Developer Audience</B>

NDAS API are designed for use by C/C++ programmers,
Familiarity with networking, NDAS Device and ATA(to use
advanced features of NDASComm) is required.



<B>Run-Time Requirements</B>

NDAS API can be used on Windows 2000, Windows XP and Windows
Server 2003 with LPX(Lean Packet eXchange) protocol
installed.

@@!!SYMREF


@@About NDASComm
This section describes NDASComm considerations, architecture,
and capabilities available to NDASComm developers.

@@NDASComm Architecture
<TITLE NDAS API Architecture>

Following drawing describes the architecture of the NDAS API.



<IMAGE ndas_architecture>



NDAS Comm library translates & packs API Call command into
Lanscsi protocol. Packed command transfered to the NDAS
Device through ethernet network via LPX protocol.



NDAS Heartbeat Listener library detects the NDAS Device by
catching heartbeat packet from the NDAS Device via ethernet.



There is 3 versions of the NDAS Device. Only ATA Device(HDD)
can be attached to NDAS Device Ver 1.0.



NDAS Device does not send any command to the host except
broadcast message which contains devices version that can be
catched by NDAS Listener.

@@LPX Protocol
\Short for Lean Packet eXchange, the suite of communication
protocols used to connect NDAS Device on the ethernet. LPX is
designed for home network by XIMETA Inc.

@@NDAS Device
NDAS(Network Direct Attached Storage) is XIMETA's patented
technology which enables all digital storage devices(HDD,
ODD, Memory, Tape Drives) connect into standard Ethernet
network and all the users or systems in the network directly
control, use and share those devices.

@@NDAS Device Version
There is 3 versions of the NDAS Device.



<B>NDAS Device Ver 1.0</B>

The NDAS Device 1.0 has ability to operate only following ATA
commands.

READ(0x20)

WRITE(0x30)

VERIFY(0x40)

IDENTIFY(0xFC)

SET FEATURE(0xEF)



Actually, the NDAS Device 1.0 does not support ATA commands
but translates few command to real ATA commands. So it is not
recommanded to use NdasCommIdeCommand fuinction with 1.0.



The NDAS Device 1.0 supports ethernet connection upto
100Mbps.

The NDAS Device 1.0 supports Multiword DMA level 0, 1 and 2.

The NDAS Device supports ATA device.

The NDAS Device does not support ATAPI device.



<B>NDAS Device Ver 1.1</B>

The NDAS Device 1.1 supports virtually all ATA commands
including packet command(but NDASComm dose not supports
packet command). The NDAS Device 1.1 receives ATA command and
send the command directly to unit device attached to the NDAS
Device.



The NDAS Device 1.1 supports ethernet connection upto
100Mbps.

The NDAS Device 1.1 supports Multiword DMA level 0, 1 and 2.

The NDAS Device supports ATA device.

The NDAS Device supports ATAPI device.



<B>NDAS Device Ver 2.0</B>

The NDAS Device 2.0 is same with 1.1 in supporting commands.
See table for detailed differences.



The NDAS Device 2.0 supports ethernet connection upto
100Mbps.

The NDAS Device 2.0 supports Multiword DMA level 0, 1 and 2.

The NDAS Device 2.0 supports Ultra DMA level 0 to 6.

The NDAS Device supports ATA device.

The NDAS Device supports ATAPI device.



<B>Differences by the version of the NDAS Device.</B>
<TABLE>
\             1.0                  1.1             2.0
------------  -------------------  --------------  -------------------
Max Speed     100MBps              100Bps          1000Bps
Unit Device   ATA Device           ATA/ATAPI       ATA/ATAPI Device
                                    Deivce          
DMA level     Multiword DMA 0, 1   Multiword DMA   Multiword DMA 0, 1
               and 2                0, 1 and 2      and 2<P>
                                                    Ultra DMA 0 to 6
Command       READ, WRITE,         All ATA         All ATA Commands
               VERIFY, IDENTIFY,    Commands        
               SET FEATURE                          
</TABLE>

@@Lanscsi Protocol
Lanscsi is a protocol designed to send/recieve
commands/replies from/to the NDAS Device.

@@Using NDASComm
The following is a step-by-step guide to getting started with
NDASComm library. It is designed to provide an understanding
\of basic NDASComm functions and data structures, and how
they work together.



<B>Creating a Basic NDASComm Application</B>

To create a basic NDASComm application

. Create a new empty project.

. Ensure that the build environment refers to the Include,
Lib and Src directories of the NDASComm library.

. Ensure that the build envionment links to the NDASComm
library file ndascomm.lib.

. Begin programming the NDASComm application. Use the
NDASComm by including the NDASComm header file.

. If the application is using DLL, ensure to place
ndascomm.dll where an NDASComm application can find at.

<U> Note</U> some other headers should be used for
convenience specially the printf() function.


<CODE>
\#include "ndascomm.h"

BOOL main() {
  printf("Starting NDAS application\\n");
  return TRUE;
}
</CODE>


<B>Initializing NDASComm</B>

All NDASComm application must be initialized to ensure that
NDASComm are supported on the system.

To Initialize NDASComm library

. Call NdasCommInitialize and check errors


<CODE>
BOOL bResult = NdasCommInitialize();
if(!bResult)
{
  printf("Error %d at NdasCommInitialize()\\n", GetLastError());
  return FALSE;
}
</CODE>


The NdasCommInitialize function is called to initiate use of
ndascomm.lib


<CODE>
DWORD dwVersion = NdasCommGetAPIVersion();
WORD wMajorVersion = LOWORD(dwVersion);
WORD wMinororVersion = HIWORD(dwVersion);
</CODE>


The NdasCommGetAPIVersion function is called to retrieve the
\version of ndascomm.lib



<B>Creating connection to an NDAS Device</B>

After initialization, a NDASCOMM_CONNECTION_INFO struct must
be initialized.


<CODE>
NDASCOMM_CONNECTION_INFO m_ci;
ZeroMemory(&m_ci, sizeof(m_ci));

m_ci.address_type = NDASCOMM_CONNECTION_INFO_TYPE_ID_A; // address using ansi code ID string
m_ci.login_type = NDASCOMM_LOGIN_TYPE_NORMAL; // operates active jobs
m_ci.UnitNo = 0; // use first unit device
m_ci.bWriteAccess = TRUE; // login with write privilage
m_ci.ui64OEMCode = NULL; // use default normal user password
m_ci.protocol = IPPROTO_LPXTCP; // connect with LPX protcol
memcpy(m_ci.sDeviceStringId, "ABCDE12345EFGHI67890", 20); // set 20 chars of ansi code ID
memcpy(m_ci.sDeviceStringKey, "12345", 5); // set 5 chars of ansi write key
</CODE>


After NDASCOMM_CONNECTION_INFO struct initialized, Connect to
the NDAS Device using NdasCommConnect.


<CODE>
HNDAS hNDASDevice = NdasCommConnect(&ConnectionInfo);

if(!hNDASDevice)
{
  printf("Error %d at NdasCommConnect()\\n", GetLastError());
  return FALSE;
}
</CODE>


<B>Retrieving informations from an NDAS Device and/or Unit
Device</B>



See examples of NdasCommGetDeviceID(),
NdasCommGetDeviceInfo(), NdasCommGetUnitDeviceInfo()



<B>Operates basic commands</B>



See examples of NdasCommBlockDeviceRead(),
NdasCommBlockDeviceWrite(), NdasCommBlockDeviceVerify(),
NdasCommSetFeatures(), NdasCommBlockDeviceWriteSafeBuffer()



<B>Operates advanced commands</B>



See examples of NdasCommIdeCommand(), NdasCommVendorCommand()



<B>Disconnecting connection from an NDAS Device</B>

The NdasCommDisconnect function is called after the use of
the NDAS Device.


<CODE>
bResult = NdasCommDisconnect(hNDASDevice);
if(!bResult)
{
  printf("Error %d at NdasCommDisconnect()\\n", GetLastError());
  return FALSE;
}
</CODE>

@@NDAS Unit Device
The NDAS Unit Device(or Unit Device) is a ATA/ATAPI device
which is attached to the NDAS Device so that a host can
access to the Device via the NDAS Device.



The number of maximum Unit Device that the NDAS Device
supports can be obtained by NdasCommGetDeviceInfo.

@@NDASComm API
This is the default topic text for new topics added with the
Doc-O-Matic editor. You can freely modify the text and
structure of new text by changing the template topics:

    * Open the Doc-O-Matic project "template.dox" located in
      your Doc-O-Matic/support directory.
    * The project contains a list of topics that you can
      freely modify.
    * You can even add your own topics and delete existing
      \ones.
    * Whatever is in the file "template.dtx" will be used as
      templates that you can choose from.


@@NDASHear API
This is the default topic text for new topics added with the
Doc-O-Matic editor. You can freely modify the text and
structure of new text by changing the template topics:

    * Open the Doc-O-Matic project "template.dox" located in
      your Doc-O-Matic/support directory.
    * The project contains a list of topics that you can
      freely modify.
    * You can even add your own topics and delete existing
      \ones.
    * Whatever is in the file "template.dtx" will be used as
      templates that you can choose from.

@@About NDASHear
This is the default topic text for new topics added with the
Doc-O-Matic editor. You can freely modify the text and
structure of new text by changing the template topics:

    * Open the Doc-O-Matic project "template.dox" located in
      your Doc-O-Matic/support directory.
    * The project contains a list of topics that you can
      freely modify.
    * You can even add your own topics and delete existing
      \ones.
    * Whatever is in the file "template.dtx" will be used as
      templates that you can choose from.

@@Using NDASHear
This is the default topic text for new topics added with the
Doc-O-Matic editor. You can freely modify the text and
structure of new text by changing the template topics:

    * Open the Doc-O-Matic project "template.dox" located in
      your Doc-O-Matic/support directory.
    * The project contains a list of topics that you can
      freely modify.
    * You can even add your own topics and delete existing
      \ones.
    * Whatever is in the file "template.dtx" will be used as
      templates that you can choose from.
