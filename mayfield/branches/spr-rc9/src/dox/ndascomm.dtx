
@@Overview
<B>Purpose</B>

NDAS Communication API (NDASCOMM) enables programmers to
interact with a NDAS device directly With NDASCOMM,
programmers can operate not only basic operations such as
reading or writing, but also advanced operations including
most parts of the ATA commands such as identifying or setting
features.

<B>Developer Audience</B>

NDAS API's are designed for use by C/C++ programmers,
Familiarity with networking, NDAS devices and ATA commands
(to use make use of advanced features of NDASCOMM) is
required.

<B>Run-time Requirements</B>

NDAS API requires Windows 2000, Windows XP or Windows Server
2003 and a system with LPX protocol and Windows Socket 2.

@@!!SYMREF

@@About NDASComm
<B>NDAS Communication API</B>

NDAS Communication API(NDASComm) provides functions to
communicate with NDAS Device(s) with LPX protocol over
Ethernet.



NDASComm does not only provide basic functions that
application can send ATA command to the NDAS Device but also
utility functions to read, write, verify or identify the NDAS
Unit Device.



<COLOR Red>This library is provided to support manufacturing
processes and for diagnostic purpose. Do not redistribute
this library.</COLOR>

@@NDASComm Architecture
<TITLE NDAS API Architecture>

Figure 1 describes the architecture of NDAS API.



<IMAGE ndas_architecture> Figure 1. Architecture of NDAS API.



NDAS Communication Library translates API calls into NDAS
device commands and sends them to a NDAS device with LPX
protocol over Ethernet.

NDAS Heartbeat Listener Library listens a designated
heartbeat packet from NDAS devices in the network and calls
the user-supplied callback function with a NDAS device
information.

@@LPX Protocol

\Short for Lean Packet eXchange Protocol.
It is defined by XIMETA, Inc. and 
used to communicate with NDAS device over the Ethernet.
It is a light-weight and efficient protocol to provide
high performance data transfer between a NDAS host and a NDAS device.

@@NDAS Device

NDAS (Network Direct Attached Storage) is XIMETA's patented
technology which enables all digital storage devices (HDD,
ODD, Memory, Tape Drives) connect into standard Ethernet
network and all the users or systems in the network directly
control, use and share those devices.

@@NDAS Device Version
Up to now, three versions of the NDAS devices are released.

<B>NDAS device 1.0</B>

The NDAS device 1.0 provides following ATA commands:

    * READ(0x20)
    * WRITE(0x30)
    * VERIFY(0x40)
    * IDENTIFY(0xFC)
    * SET FEATURE(0xEF)

Internally, NDAS device 1.0 does not support direct ATA
commands, but it translates pre-defined command set into
actual ATA comands. It is not recommanded to use
NdasCommIdeCommand function with NDAS device 1.0.

Characteristics of NDAS device 1.0 are as follows:

    * supports ethernet connection up to 100 Mbps.
    * supports Multiword DMA level 0, 1 and 2.
    * supports ATA devices.
    * does not support ATAPI devices.

<B>NDAS Device Ver 1.1</B>

NDAS Device 1.1 supports virtually all ATA commands including
packet commands (but NDASCOMM dose not supports packet
command at this time). The NDAS Device 1.1 receives an ATA
command and send it directly to the NDAS unit device.

Characteristics of NDAS device 1.1 are as follows:

    * supports ethernet connection up to 100 Mbps.
    * supports Multiword DMA level 0, 1 and 2.
    * supports ATA devices.
    * supports ATAPI devices.



<B>NDAS Device Ver 2.0</B>

NDAS Device 2.0 has a same ATA command set as NDAS device
1.1.

See the table for differences.

Characteristics of NDAS device 2.0 are as follows:

    * supports ethernet connection up to 1 Gbps.
    * supports Multiword DMA level 0, 1 and 2.
    * supports Ultra DMA level 0 to 6.
    * supports ATA devices.
    * supports ATAPI devices.



<B>Differences by the version of the NDAS Device.</B>
<TABLE>
\             1.0                  1.1             2.0
------------  -------------------  --------------  -------------------
Max Speed     100 MBps             100 MBps        1 GBps
Unit Device   ATA Device           ATA/ATAPI       ATA/ATAPI Device
                                    Deivce          
DMA level     Multiword DMA 0, 1   Multiword DMA   Multiword DMA 0, 1
               and 2                0, 1 and 2      and 2<P>
                                                    Ultra DMA 0 to 6
Command       READ, WRITE,         All ATA         All ATA Commands
               VERIFY, IDENTIFY,    Commands        
               SET FEATURE                          
</TABLE>

@@Lanscsi Protocol
Lanscsi is a protocol designed to send/recieve
commands/replies from/to the NDAS Device.

@@Using NDASComm
The following is a step-by-step guide to getting started with
NDASCOMM library. It is designed to provide an understanding
\of basic NDASComm functions and data structures, and how
they work together.



<B>Creating a Basic NDASCOMM Application</B>



To create a basic NDASCOMM application

      1. Create a new empty project.
      2. Set the build environment to include paths for the
         headers and libraries.
      3. Set link options to link against ndascomm.lib.
      4. Include "ndascomm.h" to use NDASCOMM API.
      5. Ensure to place ndascomm.dll in an NDASCOMM
         application's search path.


<CODE>

 \#include \<ndascomm.h\>

 int main()
 {
   printf("Starting NDAS application\\n");
   return 0;
 }

</CODE>


<B>Initializing NDASComm</B>



All NDASComm applications should initialize NDASCOMM library
first before calling other API functions to ensure that
NDASComm is supported on the system.

To Initialize NDASComm library

    * Call NdasCommInitialize and check errors


<CODE>

 BOOL bResult = NdasCommInitialize();
 if (!bResult)
 {
   printf("Error %d at NdasCommInitialize()\\n", GetLastError());
   return FALSE;
 }

</CODE>


The NdasCommInitialize function is called to initiate use of
ndascomm.lib


<CODE>

 DWORD dwVersion = NdasCommGetAPIVersion();
 WORD wMajorVersion = LOWORD(dwVersion);
 WORD wMinororVersion = HIWORD(dwVersion);

</CODE>


The NdasCommGetAPIVersion function is called to retrieve the
\version of ndascomm.lib



<B>Creating connection to an NDAS Device</B>



After initialization, a NDASCOMM_CONNECTION_INFO struct must
be initialized.


<CODE>

 NDASCOMM_CONNECTION_INFO m_ci;
 ZeroMemory(&m_ci, sizeof(NDASCOMM_CONNECTION_INFO));

 /* address using ansi code ID string */
 m_ci.address_type = NDASCOMM_CONNECTION_INFO_TYPE_ID_A;

 /* operates active jobs */
 m_ci.login_type = NDASCOMM_LOGIN_TYPE_NORMAL;

 /* use first unit device */
 m_ci.UnitNo = 0;

 /* login with write privilage */
 m_ci.bWriteAccess = TRUE;

 /* use default normal user password */
 m_ci.ui64OEMCode = NULL;

 /* connect with LPX protcol */
 m_ci.protocol = IPPROTO_LPXTCP;

 /* set 20 chars of ansi code ID */
 CopyMemory(m_ci.sDeviceStringId, "ABCDE12345EFGHI67890", 20);

 /* set 5 chars of ansi write key */
 CopyMemory(m_ci.sDeviceStringKey, "12345", 5);

</CODE>


After NDASCOMM_CONNECTION_INFO struct initialized, Connect to
the NDAS Device using NdasCommConnect.


<CODE>

 HNDAS hNDASDevice = NdasCommConnect(&ConnectionInfo);

 if(!hNDASDevice)
 {
   printf("Error %d at NdasCommConnect()\\n", GetLastError());
   return FALSE;
 }

</CODE>


<B>Retrieving informations from an NDAS Device and/or Unit
Device</B>



See examples of NdasCommGetDeviceID(),
NdasCommGetDeviceInfo(), NdasCommGetUnitDeviceInfo()



<B>Operates basic commands</B>



See examples of NdasCommBlockDeviceRead(),
NdasCommBlockDeviceWrite(), NdasCommBlockDeviceVerify(),
NdasCommSetFeatures(), NdasCommBlockDeviceWriteSafeBuffer()



<B>Operates advanced commands</B>



See examples of NdasCommIdeCommand(), NdasCommVendorCommand()



<B>Disconnecting connection from an NDAS Device</B>



The NdasCommDisconnect function is called after the use of
the NDAS Device.


<CODE>

 bResult = NdasCommDisconnect(hNDASDevice);
 if(!bResult)
 {
   printf("Error %d at NdasCommDisconnect()\\n", GetLastError());
   return FALSE;
 }

</CODE>

@@NDAS Unit Device


NDAS unit device (or unit device) is an ATA/ATAPI device
which is attached to a NDAS device so that a host can access
to the device via NDAS device.



The number of maximum unit devices that a NDAS device
supports can be obtained by NdasCommGetDeviceInfo.

@@NDASComm API
<TITLE NDAS Communication API>

This is the default topic text for new topics added with the
Doc-O-Matic editor. You can freely modify the text and
structure of new text by changing the template topics:

    * Open the Doc-O-Matic project "template.dox" located in
      your Doc-O-Matic/support directory.
    * The project contains a list of topics that you can
      freely modify.
    * You can even add your own topics and delete existing
      \ones.
    * Whatever is in the file "template.dtx" will be used as
      templates that you can choose from.


